\documentclass[acmsmall]{acmart}

\setcopyright{rightsretained}
\copyrightyear{2021}
\acmYear{2021}
\acmConference[HATRA '21]{Human Aspects of Types and Reasoning Assistants}{October 2021}{Chicago, IL}

\begin{document}

\title{The Goals of the Luau Type System}

\author{Andy Friesen}
\author{Alan Jeffrey}
\author{Other People?}
\affiliation{
  \institution{Roblox}
  \city{San Mateo}
  \state{CA}
  \country{USA}
}

\begin{abstract}
  A position paper about the goals Luau type system.
\end{abstract}

\maketitle

\section{Introduction}

The Roblox~\cite{Roblox} platform allows anyone to create shared,
immersive, 3D experiences.  At the time of writing, there are
approximately eight million experiences available on Roblox, created
by eight million developers.  Roblox developers are often young, for
example there are over 200 Roblox coding camps in over 65 countries
listed at~\cite{AllEducators}.

The Luau programming language~\cite{Luau} is the scripting language
used by developers of Roblox experiences. Luau is derived from the Lua
programming language~\cite{Lua}, with additional capabilities,
including a type inference engine.

This paper will discuss some of the goals of the Luau type system, and
why those goals are slightly different from other type systems.

\section{Infallible types}
\section{Strict types}
\section{Nonstrict types}
\section{Conclusions}

\bibliographystyle{ACM-Reference-Format} \bibliography{bibliography}

\end{document}
