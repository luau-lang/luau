\documentclass[acmsmall]{acmart}

\setcopyright{rightsretained}
\copyrightyear{2021}
\acmYear{2021}
\acmConference[HATRA '21]{Human Aspects of Types and Reasoning Assistants}{October 2021}{Chicago, IL}
\acmBooktitle{HATRA '21: Human Aspects of Types and Reasoning Assistants}

\newcommand{\squnder}[1]{\underline{#1}}
\newcommand{\infer}[2]{\frac{\textstyle#1}{\textstyle#2}}
\newcommand{\erase}{\mathrm{erase}}
\newcommand{\evCtx}{\mathcal{E}}
\newcommand{\NIL}{\mathsf{nil}}
\newcommand{\TRUE}{\mathsf{true}}
\newcommand{\FALSE}{\mathsf{false}}
\newcommand{\ERROR}{\mathsf{error}}
\newcommand{\IF}{\mathsf{if}\,}
\newcommand{\THEN}{\,\mathsf{then}\,}
\newcommand{\ELSE}{\,\mathsf{else}\,}
\newcommand{\END}{\,\mathsf{end}}
\newcommand{\FUNCTION}{\mathsf{function}\,}
\newcommand{\RETURN}{\mathsf{return}\,}

\begin{document}

\title{Position Paper: Some Goals of the Luau Type System}

\author{Andy Friesen}
\author{Alan Jeffrey}
\author{Other People?}
\affiliation{
  \institution{Roblox}
  \city{San Mateo}
  \state{CA}
  \country{USA}
}

\begin{abstract}
  Luau is the scripting language used in creating Roblox experiences.
  It is a typed language based on the dynamically-typed Lua language,
  and uses type inference to infer types. These types are then in the
  IDE, for example when providing autocomplete suggestions. In this
  paper, we describe some of the goals of the Luau type system,
  focusing on where the goals are different from those of other type systems.
\end{abstract}

\maketitle

\section{Introduction}

The Roblox~\cite{Roblox} platform allows anyone to create shared,
immersive, 3D experiences.  At the time of writing, there are
approximately 20~million experiences available on Roblox, created
by 8~million developers.  Roblox creators are often young, for
example there are over 200~Roblox kids' coding camps in 65~countries
listed at~\cite{AllEducators}.

The Luau programming language~\cite{Luau} is the scripting language
used by developers of Roblox experiences. Luau is derived from the Lua
programming language~\cite{Lua}, with additional capabilities,
including a type inference engine.

This paper will discuss some of the goals of the Luau type system,
focusing on where the goals are different from those of other type systems.

\section{Human Aspects}
\subsection{Heterogenous developer community}

Quoting a Roblox 2020 report \cite{RobloxDevelopers}:
\begin{itemize}
\item Adopt Me! now has over 10 billion plays and surpassed 1.6 million concurrent users in game earlier this year.
\item Piggy, launched in January 2020, has close to 5 billion visits in just over six months.
\item There are now 345,000 developers on the platform who are monetizing their games.
\end{itemize}
This demonstrates how heterogenous the Roblox developer community is:
developers of experiences with plays measured in billions are on the same
platform as children first learning to code. Moreover, \emph{both of
these groups are important}, as the professional development studios
bring high-quality experiences to the platform, and the beginning creators
contribute to the energetic creative community.

\subsection{Goal-driven learning}

All developers are goal-driven, but this is especially true for
learners. A learner will download Roblox Studio (the IDE) with an
experience in mind, often designing an obstacle course (an ``obby'')
to play in with their friends.

The user experience of developing a Roblox experience is primarily a
3D interactive one, seen in Fig.~\ref{fig:studio}(a). The user designs
and deploys 3D assets such as terrain, parts and joints, and provides
them with physics attributes such as mass and orientation. The user
can interact with the experience in Studio, and deploy it to a Roblox
server so anyone with the Roblox app can play it.

\begin{figure}
\includegraphics[width=0.48\textwidth]{studio-mow.png}
\includegraphics[width=0.48\textwidth]{studio-script-editor.png}
\caption{Roblox Studio's 3D environment editor (a), and script editor (b)}
\label{fig:studio}
\end{figure}

At some point during experience design, the user of Studio has a need
which can't be met by the physics engine alone. ``The stairs should
light up when a player walks on them'' or ``a firework is set off
every few seconds.'' At this point they will discover the script
editor, seen in Fig.~\ref{fig:studio}(b), and the Luau programming language.

This onboarding experience is different from many initial exposures to
programming, in that by the time the user first opens the script
editor, they have already built much of their creation, and have a
very specific concrete aim.  It suggests a Luau goal for helping the
majority of creators: \emph{support learning how to perform specific
tasks} (for example through autocomplete suggestions and
documentation).

\subsection{Type-driven development}

Professional development studios are also goal-directed (though the
goals may be more abstract, such as ``decrease user churn'' or
``improve frame rate'') but have needs that are less common in
learners:
\begin{itemize}

\item \emph{Code planning}:
  code spends much of its development time in an incomplete state,
  with holes that will be filled in later.

\item \emph{Code refactoring}:
  experiences evolve over time, and it easy for changes to
  break previously-held invariants.

\item \emph{Defect detection}:
  code has errors, and detecting these at runtime (for example by crash telemetry)
  can be expensive and recovery can be time-consuming.
  
\end{itemize}
Detecting defects ahead-of-time is a traditional goal of type systems,
resulting in an array of techniques for establishing safety results,
surveyed for example in~\cite{TAPL}. Supporting code planning and
refactoring are some of the goals of \emph{type-driven
development}~\cite{TDDIdris} under the slogan ``type, define,
refine''.  For example. a common use of type-driven development is to
rename a property, which is achieved by changing the name in one place,
and then fixing the resulting type errors---once the type system stops
reporting errors, the refactoring is complete.

To help support the transition from novice to experienced developer,
types are introduced gradully, through API documentation and type discovery.
Type inference provides many of the benefits of type-driven development
even to creators who are not explicitly providing types.

\section{Types}
\subsection{Infallible types}

Goal: \emph{support type-driven tools in all programs}.

Programs spend much of their time under development in an incomplete state, even if the final arifact
is well-typed. Type-driven tools should support this, by providing type information for all programs.
An analogy is infallible parsers, which perform error recovery and provide an AST for all input texts.

Program analysis can still flag type errors, for example with red
squiggly underlining. Formalizing this, rather than a judgement
$\Gamma\vdash M:T$, for an input terms $M$, there is a judgement
$\Gamma \vdash M \Rightarrow M' : T$ where $M'$ is an output term
where some subterms are flagged $\squnder{M}$. Write $\erase(M)$
for the result of erasing flagged type errors: $\erase(\squnder{M}) = \erase(M)$.

%% For example the usual
%% type rules for field access becomes:
%% \[
%%   \infer{
%%     \Gamma \vdash M \Rightarrow M' : T
%%   }{
%%     \Gamma \vdash M.\ell \Rightarrow M'.\ell : U
%%   }
%%   [
%%     T = \{ \overline{\ell:U} \} \mbox{ and } (\ell:U) \in (\overline{\ell:U})
%%   ]
%% \]
%% but there is also a rule for unsuccesful field access:  
%% \[
%%   \infer{
%%     \Gamma \vdash M \Rightarrow M' : T
%%   }{
%%     \Gamma \vdash M.\ell \Rightarrow \squnder{M'.\ell} : U
%%   }
%%   [
%%     T = \{ \overline{\ell:U} \} \mbox{ implies } \ell \not\in \overline{\ell}
%%   ]
%% \]
%% In this type rule, $U$ is unconstrained.

The goal of infallible types is that every term can be typed:
\begin{itemize}
\item \emph{Typability}: for every $M$ and $\Gamma$,
  there are $M'$ and $T$ such that $\Gamma \vdash M \Rightarrow M' : T$.
\item \emph{Erasure}: if $\Gamma \vdash M \Rightarrow M' : T$
  then $\erase(M) = \erase(M')$ 
\end{itemize}
Some issues raised by infallible types:
\begin{itemize}
\item Which heuristics should be used to provide types for flagged programs? For example, could one
  use minimal edit distance to correct for spelling mistakes in field names?
\item How can we avoid cascading type errors, where a developer is
  faced with type errors that are artifacts of the heuristics rather
  than genuine errors?
\item How can the goals of an infallible type system be formalized?
\end{itemize}
Related work: lots on type error reporting~\cite{???}, and on
heuristics for program repair~\cite{???}, but not type repair, or on
the semantics of programs with type errors.

\subsection{Strict types}

Goal: \emph{no false negatives.}

For developers who are interested in defect detection, Luau provides a \emph{strict mode},
which acts much like a traditional, sound, type system. This has the goal of ``no false negatives'' that is any
run-time error is flagged.

The usual presentation of type safety is using type preservation and
progress. This requires:
\begin{itemize}
\item \emph{Operational semantics}: a reduction judgement $M \rightarrow N$ on terms.
\item \emph{Values}: a subset of terms representing a successfully completed evaluation.
\end{itemize}
We then represent error states as stuck states (terms that are not
values but cannot reduce), and showing that no well-typed program is
stuck. This is not true if typing is infallible, but can fairly
straightforwardly be adapted:
\begin{itemize}
\item \emph{Progress}: if ${} \vdash M \Rightarrow M'$, then either $M \rightarrow N$ or $M$ is a value or $M'$ is flagged.
\item \emph{Preservation}: if ${} \vdash M \Rightarrow M'$ and $M \rightarrow N$ and ${} \vdash N \Rightarrow N'$ and $N'$ is flagged, then $M'$ is flagged.
\end{itemize}
Some issues raised by infallible types:
\begin{itemize}
\item How should the judgements and their metatheory be set up?
\item How should generic functions be handled?
\item What does type inference or bidirectional typechecking look like in this setting?
\end{itemize}
Related work: blame analysis~\cite{???}.

\subsection{Nonstrict types}

Goal: \emph{no false positives.}

For developers who are not interested in defect detection, type-driven
tools such as autocomplete can still be useful, and type-directed
development can still be useful. For such developers, Luau provides a
\emph{nonstict mode}, which we hope will eventually be useful for all
developers. This does \emph{not} aim for soundness, but instead has
the goal of ``no false positives``, in the sense that any flagged code
is guaranteed to produce a runtime error when executed.

On the face of it, this is undecidable, since a program such as
$(\IF f() \THEN \ERROR \END)$ will produce a runtime error when $f()$ is
$\TRUE$, but we can aim for a weaker property, thjat all flagged code
is either dead code or will produce an error. Either of these is a
defect, so deserves flagging, even if the tool does not know
which. For example, using this definition
$(\IF f() \THEN \squnder{\ERROR} \END)$ is not a false positive, but
$(\squnder{\IF f() \THEN \ERROR \END})$ might be.

We can formalize this by defining an \emph{evaluation context}
$\evCtx[\bullet]$, defining a \emph{redex} of $M$ to be a subterm $N$
such that $M = \evCtx[N]$, and defining $M{\Uparrow}$ when there is no
value $V$ such that $M \rightarrow^* V$. We can then define:

\begin{itemize}
\item \emph{Snappy name}: if ${} \vdash M \Rightarrow M'$ and $M'$ has a flagged redex
  then $M{\Uparrow}$.
\end{itemize}
Some issues raised by nonstrict types:
\begin{itemize}

\item Under this definition, any function that will terminate is unflagged, so
  flagging will often move from function definitions to call sites.

\item This definition will not allow an unchecked use of an optional value
  to be flagged, for example if we define $\FUNCTION g()\,\IF f() \THEN \RETURN 5 \ELSE \RETURN \NIL \END$
  then a strict type system can flag $1 + g()$ but a nonstrict one cannot.

\item Property update of tables in languages like Luau always succeeds
  (the property is inserted if it did not exist), which interacts
  badly with compositional analysis. For example, if module $A$
  exports a table $t=\{p=5\}$, module $B$ assumes $t.p$ is a number,
  and module $C$ sets $t.p=\FALSE$, then we cannot flag $C$ since the
  assignment will succeed, and we cannot flag $A$ or $B$ without some
  whole-program analysis.

\item The natural formulation of function types in a nonstrict setting
  is that of~\cite{???}: if $f: T \rightarrow U$ and $f(v) \rightarrow^* w$
  then $v:T$ and $w:U$. This formulation is \emph{covariant} in $T$,
  not \emph{contavariant}.
  
\end{itemize}
Related work: success types~\cite{???} and incorrectness logic~\cite{???}.

\subsection{Mixing types}

Goal: \emph{support mixed strict/nonstrict development}.

Like every active software community, Roblox developers share code
with one another constantly.  First- and third-party developers alike
frequently share entire software packages written in Luau.  To add to
this, many Roblox games are authored not by just one developer, but a
whole team.

It is therefore crucial that we offer first-class support for mixing
code written in strict and nonstrict modes.

Some issues raised by mixed-mode types:
\begin{itemize}

\item How much feedback can we offer for a nonstrict script that is
  importing strict-mode code?

\item In strict mode, how do we talk about values and types that are
  drawn from nonstrict code?

\item How can we combine the goals of strict and nonstrict types?

\item Can we have strict and non-strict mode infer the same types,
  only with different flagging?

\end{itemize}
Related work: none???

\section{Conclusions}

In this paper, we have presented some of the goals of the Luau type
system, and how they map to the needs of the Roblox creator
community. We have sketched what a solution might look like, all that
remains is to draw the damn owl~\cite{owl}.

\bibliographystyle{ACM-Reference-Format} \bibliography{bibliography}

\end{document}
